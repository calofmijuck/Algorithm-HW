\documentclass[12pt]{article}
\usepackage{geometry}
\usepackage{xcolor}
\usepackage{listings}
\usepackage{graphicx}
\geometry{
	top = 20mm,
	left = 25mm,
	right = 25mm,
	bottom = 20mm
}
\geometry{a4paper}
%\renewcommand{\familydefault}{\sfdefault}
\renewcommand{\baselinestretch}{1.3}

\definecolor{mGreen}{rgb}{0,0.6,0}
\definecolor{mGray}{rgb}{0.5,0.5,0.5}
\definecolor{mPurple}{rgb}{0.58,0,0.82}
\definecolor{mRed}{rgb}{1, 0, 0}
\definecolor{backgroundColour}{rgb}{0.95,0.95,0.92}

\lstdefinestyle{CStyle}{
	commentstyle=\color{mGreen},
	keywordstyle=\color{mPurple},
	numberstyle=\tiny\color{mGray},
	stringstyle=\color{mGreen},
	basicstyle=\ttfamily\footnotesize,
	emph = {},
	emphstyle=\color{red},
	breakatwhitespace=false,         
	breaklines=true,                 
	captionpos=b,                    
	keepspaces=true,                 
	numbers=left,                    
	numbersep=10pt,                  
	showspaces=false,                
	showstringspaces=false,
	showtabs=false,                  
	tabsize=4,
	language=C++,
}

\lstdefinestyle{PyStyle}{
	commentstyle=\color{mGreen},
	keywordstyle=\color{mPurple},
	numberstyle=\tiny\color{mGray},
	stringstyle=\color{mGreen},
	basicstyle=\ttfamily\footnotesize,
	emph = {},
	emphstyle=\color{red},
	breakatwhitespace=false,         
	breaklines=true,                 
	captionpos=b,                    
	keepspaces=true,                 
	numbers=left,                    
	numbersep=-10pt,                  
	showspaces=false,                
	showstringspaces=false,
	showtabs=false,                  
	tabsize=4,
	language=Python,
}

%opening
\title{\sffamily Algorithms: Strongly Connected Components}
\author{2017-18570 Sungchan Yi\\Dept. of Computer Science and Engineering}
\date{May 15th, 2019}
\begin{document} 

\maketitle
\tableofcontents

\section{Introduction}
In this homework, we implement a program that finds the strongly connected components (SCC) of a given directed graph using Kosaraju's Algorithm with C++ language. We must be able to find the SCCs when the directed graph is given in a form of adjacency matrix, or adjacency list.
The following are the restrictions on given inputs.
\begin{itemize}
	\item First line contains number of vertices $V$ and number of edges $E$.
	\item The next $E$ lines will contain information about edges. 
	\item It will be given as $u$ $v$, denoting that there exists a directed edge from $u$ to $v$.
\end{itemize}
The program should output each SCC in a line such that
the vertices in each SCC are sorted, and the lines appear in the output are in lexicographic order.

\section{Implementation}
The algorithm was implemented similarly to the algorithm in the textbook.\\
\\
\textsc{Strongly-Connected-Components}($G$) (Kosaraju)
\begin{enumerate}
	\item Call \textsc{DFS}($G$) and compute finishing times $u.f$ for each vertex $u$.
	\item Compute $G^T$.
	\item Call \textsc{DFS}($G^T$), but consider the vertices in order of decreasing $u.f$.
	\item Sort and output the vertices of each tree in the depth-first forest formed in 3 as a separate SCC
\end{enumerate}
\textsc{DFS}$(G)$ and computing $G^T$ takes $O(V+E)$ time. Thus the overall complexity is $O(V+E)$. (Without considering the sorting time for each SCC)

\section{Experiments}
\subsection{Enviornment}
The following is the test environment.
\begin{itemize}
	\item OS: Ubuntu 18.04.2 LTS
	\item CPU: Intel® Core™ i5-6200U CPU @ 2.30GHz $\times$ 4
	\item g++: 7.4.0
	\item RAM: 8GB  
\end{itemize} 
Here is how the test was done. Run the algorithm, measure the running time without the sorting of vertices and SCCs.

\subsection{Results}
\subsubsection{Adjacency List Implementation}
\begin{center}
	\begin{tabular}{|c|c|c|c|}
		\hline
		Test File & $V$ & $E$ & Running Time ($\mu s$) \\ \hline
		\texttt{in1.txt} & 250 & 600 & 343\\ \hline 
		\texttt{in2.txt} & 500 & 1200 & 660 \\ \hline 
		\texttt{in3.txt} & 750 & 1800 & 1021 \\ \hline 
		\texttt{in4.txt} & 1000 & 2400 & 1350 \\ \hline 
	\end{tabular}
\end{center}

\subsubsection{Adjacency Matrix Implementation}
\begin{center}
	\begin{tabular}{|c|c|c|c|}
		\hline
		Test File & $V$ & $E$ & Running Time ($\mu s$) \\ \hline
		\texttt{in1.txt} & 250 & 600 & 3545\\ \hline 
		\texttt{in2.txt} & 500 & 1200 & 12780 \\ \hline 
		\texttt{in3.txt} & 750 & 1800 & 30825 \\ \hline 
		\texttt{in4.txt} & 1000 & 2400 & 63320 \\ \hline 
	\end{tabular}
\end{center}


\section{Analysis}
Consider running time $y$ as $\alpha V + \beta E + \gamma$, and use least squares approximation.
Here is the python code for the approximation.
\begin{lstlisting}[style=Pystyle]
	import numpy as np
	
	# A is fixed
	a = np.array([
		[250, 600, 1], [500, 1200, 1], [750, 1800, 1], [1000, 2400, 1]
	])
	
	# Running Time
	b = np.array([
		[343], [660], [1021], [1350]
	])
	
	# Moore-Penrose Inverse
	aplus = np.linalg.pinv(a)
	
	# Calculate least-square minimum length solution
	xplus = np.matmul(aplus, b)
	
	print(xplus)
	
	# Calculate R^2 value
	bbar = np.mean(b)
	sstot = np.linalg.norm(b - bbar) ** 2
	ssres = np.linalg.norm(np.matmul(a, xplus) - b) ** 2
	r2 = 1 - ssres/sstot
	print(r2)
\end{lstlisting}

\subsection{Adjacency List Implementation}
For the adjacency list implementation, we got $$y = 0.2001 V + 0.4803 E - 2$$
with $R^2 =0.9994$
\subsection{Adjacency Matrix Implementation}
For the matrix implementation, we got $$y = 11.6787 V + 28.0289 E - 21725$$
with $R^2 = 0.9344$. But this seemed incorrect, since it takes $O(V^2)$ time when transposing the graph. So we modified the prediction to be $O(V^2+E)$, and got $$y = 0.0930 V^2 - 15.5633 E + 7350 $$ with $R^2 = 0.9992$. Another thing to check is the negative coefficient in $-15.5633E$. This implies that the running time is faster when there are more edges. But this problem is negligible, since the test cases had $E \leq V^2$.\footnote{And thus $V^2$ is the leading term, and we should have run the least squares approximation for $y = \alpha V^2$. The negative coefficients for lower order terms do not change the overall time complexity.} Running more tests would give a correct result. 

\section{Conclusion}
For this assignment, we implemented an algorithm to find the strongly connected components using Kosaraju's Algorithm. The adjacency list implementation had time complexity $O(V+E)$ as expected, and matrix implementation had $O(V^2+E)$ as time complexity, also as expected.

\end{document}

